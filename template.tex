\documentclass[
  ignorenonframetext,
]{beamer}
\usepackage[utf8]{inputenc}
\usetheme{Oxygen}

\usepackage{movie15}
\usepackage{datetime}
\usepackage{textpos}
\usepackage{fancybox}
\usepackage{colortbl}
\usepackage{rotating}
\usepackage{multicol}
\usepackage{multirow}
\usepackage{amsmath,amssymb,amsfonts,amsbsy,dsfont, cancel}
\usepackage{mathrsfs}
\usepackage{amsthm}
\usepackage{bm}
\usepackage{array}
\usepackage[caption=false]{subfig}
%\usepackage{natbib}

%\bibliographystyle{apalike}

\newcolumntype{L}[1]{>{\raggedright\let\newline\\\arraybackslash\hspace{0pt}}m{#1}}
\newcolumntype{C}[1]{>{\centering\let\newline\\\arraybackslash\hspace{0pt}}m{#1}}
\newcolumntype{R}[1]{>{\raggedleft\let\newline\\\arraybackslash\hspace{0pt}}m{#1}}
\newcommand{\Real}{\mathbb{R}}
\newcommand{\N}{\mathbb{N}}%shrinkage operator
\newcommand{\C}{\mathbb{C}}
\newcommand{\bx}{\bm{x}}
\newcommand{\bX}{\bm{X}}
\providecommand{\ve}[1]{{\bm {#1}}} %
\providecommand{\mat}[1]{{\bm {#1}}} %
\providecommand{\est}[1]{{\widetilde {#1}}}
\newcommand{\intinf}{\int\limits_{-\infty}^{\infty}}
\newcommand{\Complex}{\mathbb{C}}
\newcommand{\To}{\longrightarrow}
\DeclareMathOperator{\en}{\negthinspace\in }
\DeclareMathOperator{\igual}{\negthinspace=\negthinspace}
\DeclareMathOperator{\x}{\negthinspace\times\negthinspace}
\providecommand{\s}[1]{\negthickspace#1\negthickspace}%
\providecommand{\am}[1]{\textcolor{blue}{#1}}
\newcommand{\Xspace}{\mathcal{X}}
\newcommand{\Hspace}{\mathcal{H}}
\newcommand{\Pmd}{\mathbb{P}}
\newcommand{\Qmd}{\mathbb{Q}}
\usefonttheme{professionalfonts}
%%%%%%%%%%%%%%%%%%%%%%%%%5
\DeclareMathAlphabet{\mathantt}{OT1}{antt}{li}{it}
\DeclareMathAlphabet{\mathpzc}{OT1}{pzc}{m}{it}
\providecommand{\abs}[1]{\lvert#1\rvert}%
\providecommand{\norm}[1]{\lVert#1\rVert} %
\providecommand{\ppunto}[2]{\langle#1,#2\rangle}% producto punto
\providecommand{\dist}[2]{{ d}(#1, #2)} % distancia
\providecommand{\promed}[1]{{\fam=8 E}\{#1\}}% operador de promedio
\providecommand{\cov}[2]{{\fam=7 cov}\{#1, #2\}}% operador de covarianza
\providecommand{\gaus}[2]{{\fam=6 N}(#1 #2)} % FDP Gauss
\providecommand{\unif}[1]{{\fam=6 U}(#1 )} % FDP Uniforme
\providecommand{\ve}[1]{{\bf {#1}}} %
\providecommand{\mat}[1]{{\bm {#1}}} %
\providecommand{\est}[1]{{\widetilde {#1}}}
\providecommand{\diag}[1]{{\rm{diag}}(#1)}% operador diagonal
\DeclareMathOperator{\subconj}{\negthinspace\subset\negthinspace }

% dummy commands to avoid "Undefined control sequence" error.
\newcommand{\inst}[1]{#1}

\providecommand{\am}[1]{\textcolor{blue}{#1}}
\providecommand{\gc}[1]{\textcolor{red}{\upshape{#1}}}


\usepackage{etoolbox}

\usepackage[nopar]{lipsum} % mock text

\newif\ifcomment
\newcommand{\Com}{\par\footnotesize\itshape\commenttrue}

\newenvironment{Itemize}
 {%
  \edef\Itemizecurrent{\the\font}%
  \itemize
  \preto{\item}{\ifcomment\par\Itemizecurrent\fi}%
 }{%
   \enditemize
 }

 \def\@email{}
 \newcommand{\email}[1]{\gdef\@email{#1}}



\providecommand{\tightlist}{%
  \setlength{\itemsep}{0pt}\setlength{\parskip}{0pt}}\usepackage{longtable,booktabs,array}
\usepackage{calc} % for calculating minipage widths
\usepackage{caption}
% Make caption package work with longtable
\makeatletter
\def\fnum@table{\tablename~\thetable}
\makeatother
\usepackage{graphicx}
\makeatletter
\newsavebox\pandoc@box
\newcommand*\pandocbounded[1]{% scales image to fit in text height/width
  \sbox\pandoc@box{#1}%
  \Gscale@div\@tempa{\textheight}{\dimexpr\ht\pandoc@box+\dp\pandoc@box\relax}%
  \Gscale@div\@tempb{\linewidth}{\wd\pandoc@box}%
  \ifdim\@tempb\p@<\@tempa\p@\let\@tempa\@tempb\fi% select the smaller of both
  \ifdim\@tempa\p@<\p@\scalebox{\@tempa}{\usebox\pandoc@box}%
  \else\usebox{\pandoc@box}%
  \fi%
}
% Set default figure placement to htbp
\def\fps@figure{htbp}
\makeatother

\makeatletter
\@ifpackageloaded{caption}{}{\usepackage{caption}}
\AtBeginDocument{%
\ifdefined\contentsname
  \renewcommand*\contentsname{Table of contents}
\else
  \newcommand\contentsname{Table of contents}
\fi
\ifdefined\listfigurename
  \renewcommand*\listfigurename{List of Figures}
\else
  \newcommand\listfigurename{List of Figures}
\fi
\ifdefined\listtablename
  \renewcommand*\listtablename{List of Tables}
\else
  \newcommand\listtablename{List of Tables}
\fi
\ifdefined\figurename
  \renewcommand*\figurename{Figure}
\else
  \newcommand\figurename{Figure}
\fi
\ifdefined\tablename
  \renewcommand*\tablename{Table}
\else
  \newcommand\tablename{Table}
\fi
}
\@ifpackageloaded{float}{}{\usepackage{float}}
\floatstyle{ruled}
\@ifundefined{c@chapter}{\newfloat{codelisting}{h}{lop}}{\newfloat{codelisting}{h}{lop}[chapter]}
\floatname{codelisting}{Listing}
\newcommand*\listoflistings{\listof{codelisting}{List of Listings}}
\makeatother
\makeatletter
\makeatother
\makeatletter
\@ifpackageloaded{caption}{}{\usepackage{caption}}
\@ifpackageloaded{subcaption}{}{\usepackage{subcaption}}
\makeatother
\ifLuaTeX
  \usepackage{selnolig}  % disable illegal ligatures
\fi
\usepackage[square,authoryear]{natbib}
\bibliographystyle{apalike}
\renewcommand\bibfont{\scriptsize}
\usepackage{bookmark}

\IfFileExists{xurl.sty}{\usepackage{xurl}}{} % add URL line breaks if available
\urlstyle{same} % disable monospaced font for URLs
\hypersetup{
  pdftitle={Template Test},
  pdfauthor={Marcos Loaiza Arias},
  pdfcreator={LaTeX via pandoc},
  unicode=true}

%%%%%%%%%%%%%%%%%%%%%%%%%%%%%%%%%%%%%%%%%%%%%%%%%%%%%%
%%%%%%%
\title[ T. Test ]{ Template Test }
\author[ M. Loaiza Arias ]{
      Marcos Loaiza Arias
  }
\date[
      November 18, 2024
  ]{
      November 18, 2024
  }
\email{
            mloaizaa@unal.edu.co
      }
\institute{
      Signal Processing and Recognition Group\par
    Universidad Nacional de Colombia\par
    Manizales, Colombia
    }
%%-------------------------------------------------------------------
%% Options
%%-------------------------------------------------------------------
\setbeamercovered{dynamic}
\setbeamertemplate{navigation symbols}{}%{footline}

\AtBeginSection[]
{
  \begin{frame}
  \frametitle{Outline}
  \tableofcontents[currentsection]
  \end{frame}
}

\begin{document}
%Test comment: Before-body.tex
\begin{frame}
    \maketitle
\end{frame}

\frame{
  \frametitle{Outline}{
  \small\tableofcontents[]
  }
}
\section{Beamer Test}\label{beamer-test}

\begin{frame}{Matplotlib Test}
\phantomsection\label{matplotlib-test}
\begin{figure}

\centering{

\pandocbounded{\includegraphics[keepaspectratio]{beamer_test_files/figure-beamer/fig-test-output-1.png}}

}

\caption{\label{fig-test}Test Figure}

\end{figure}%
\end{frame}

\section{Second Test}\label{second-test}

\begin{frame}{The text test}
\phantomsection\label{the-text-test}
This test consist in adding text.

We can also add some inline equations such as \(y = mx + b\) or centered
such as the one presented in Equation~\ref{eq-test-equation}.

\begin{equation}\phantomsection\label{eq-test-equation}{
    y = exp\left(\frac{||x-y||_2^2}{\sigma^2}\right)
}\end{equation}
\end{frame}

\begin{frame}{Text and Images}
\phantomsection\label{text-and-images}
\begin{figure}

\centering{

\pandocbounded{\includegraphics[keepaspectratio]{beamer_test_files/figure-beamer/fig-test2-output-1.png}}

}

\caption{\label{fig-test2}Test Figure}

\end{figure}%

Some text below alongside some references (\citet{qiang2022attcat},
\citet{murray2006pepsiman})
\end{frame}

\begin{frame}{Side by side}
\phantomsection\label{side-by-side}
\begin{columns}[T]
\begin{column}[c]{0.5\linewidth}
Now we add text to the left and a figure to the right
\end{column}

\begin{column}[c]{0.5\linewidth}
\begin{figure}

\centering{

\pandocbounded{\includegraphics[keepaspectratio]{beamer_test_files/figure-beamer/fig-test3-output-1.png}}

}

\caption{\label{fig-test3}Test Figure}

\end{figure}%
\end{column}
\end{columns}
\end{frame}

\begin{frame}{Side by side}
\phantomsection\label{side-by-side-1}
\begin{columns}[T]
\begin{column}[c]{0.5\linewidth}
\begin{figure}

\centering{

\pandocbounded{\includegraphics[keepaspectratio]{beamer_test_files/figure-beamer/fig-test4-output-1.png}}

}

\caption{\label{fig-test4}Test Figure}

\end{figure}%
\end{column}

\begin{column}[c]{0.5\linewidth}
Now we add text to the right and a figure to the left
\end{column}
\end{columns}
\end{frame}

\section{Final Test}\label{final-test}

\begin{frame}{Everything in Python}
\phantomsection\label{everything-in-python}
\begin{figure}

\centering{

\pandocbounded{\includegraphics[keepaspectratio]{beamer_test_files/figure-beamer/fig-test5-output-1.png}}

}

\caption{\label{fig-test5}Test Figure}

\end{figure}%

Now we add both text and figure using python code
\end{frame}

%%%-------------------------------------------------------------------
%%thank you
%%%-------------------------------------------------------------------
\begin{frame}
    \begin{center}
        {\Huge{\textbf{\textcolor[rgb]{0.00,0.00,1.00}{Thank you!}}}}\\
        \vspace{1cm}
        \insertauthor\\
                                    \texttt{email:\@email}
                        \end{center}
\end{frame}
    \begin{frame}[allowframebreaks]
        \frametitle{References}
            {\bibliography{biblio.bib
                }
            }
    \end{frame}

\end{document}
